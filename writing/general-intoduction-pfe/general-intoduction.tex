In this article we are going to present our web-based diagnosis system for melanoma skin cancer, but before diving into the technical stuff we need to define the background first, so we don't get lost in medical/technical terms later on. That is why firstly in the general medical information chapter we defined and explained some medical and biological concepts we talked about skin anatomy, the background of cancers in general and ended it talking about skin cancer in particular, its definition, most widespread types, causes, risk factors and potential treatments with (early diagnosis + removal of tumor) being the best combination for treatment. Secondly in the general AI information chapter we explained and presented Artificial Intelligence concepts, we talked about its 2 sub categories machine learning and deep learning and their characteristics and ended it talking about Computer Vision because it was the most used branch of AI in the discussed methods, we mentioned its implementation in both machine learning and deep learning and its famous applications.

In the next chapter we talk about our chosen CNN architecture for this task, we talked about our proposed model in detail, dataset we have used to train it, the preprocessing issues and techniques we have used and finally, we presented the experimental results with the accuracy being 97\%.

After that we dive in the technical stuff, where we explain the implementation process in detail, we presented our design and our application and the different software and hardware used to reach our final product.

