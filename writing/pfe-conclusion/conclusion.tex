Skin cancer is one of the most dangerous and widespread cancers in the world, it can occur to anyone of any age and of any race, just by standing in the sun for too long will increase your chances of getting it. That is why an early diagnosis will save your life. Here where the CAD's (computer aided diagnosis) systems would play an important role by implementing machine learning algorithms that could recognize, detect and classify various skin lesions. Advancements in both machine learning and deep learning have produced a lot of models that reach an accuracy above that of an expert dermatologist, but we can't do without the expertise of specialized doctors.

In this article we have presented a diagnosis system that could facilitate the process of detection and early discovery of melanoma skin cancer, it is web-based, easy to use and accessible to everyone, all you need is internet connection, a laptop or smartphone, and you are set to go. It can either be used by normal users who think that they might have melanoma, they can check that using the app before actually going to a doctor and do invasive tests (such as biopsy), or it can be used by doctors to facilitate the process of diagnosis and improve patients experience.





