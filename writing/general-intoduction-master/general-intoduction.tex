In this article we are going to do a recap and a comparative study of methods used for skin cancer detection and classification, but before diving into the technical stuff we need to define the background first so we dont get lost in medical/technical terms later on. That is why firstly in the general medical information chapter we defined and explained some medical and biological concepts we talked about skin anatomy, the background of cancers in general and ended it talking about skin cancer in particular, its definition, most wide spread types, causes, risk factors and potential treatements with (early diagnosis + removal of tumor) being the best combination for treatement. Secondly in the general AI information chapter we explained and presented Artificial Intelligence concepts, we talked about its 2 sub categories machine learning and deep learning and there caracteristics and ended it talking about Computer Vision because it was the most used branch of AI in the discussed methods, we mentioned its implementation in both machine learning and deep learning and its famous applications

reaching the state of the art, we did a recap and summary of used methods explainning there line of reasoning, datasets used, issues they faced, different feature extraction techniques and there algorithms with there evaluation metrics. Since there is alot of work in this field we firstly presented articles that impliment the usual method (image -$>$ process -$>$ train -$>$ predict) then secondly concentrated on papers that presented new ideas such as using a different type of dataset (such as Raman spectroscopy or images + clinical information) and finally we presented a comparaison table of said methods and added 2 more tables from previous researchers that did a comprehensive survey of letirature of the last few years

After the recap we discussed the differences and presented and extracted usefull information that could help futur researchers and contributors to have an overview and a ``map" of this field before diving into it. 