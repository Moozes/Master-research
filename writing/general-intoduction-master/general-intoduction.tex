In this article we are going to do a recap and a comparative study of methods used for skin cancer detection and classification, but before diving into the technical stuff we need to define the background first so we don't get lost in medical/technical terms later on. That is why firstly in the general medical information chapter we defined and explained some medical and biological concepts we talked about skin anatomy, the background of cancers in general and ended it talking about skin cancer in particular, its definition, most widespread types, causes, risk factors and potential treatments with (early diagnosis + removal of tumor) being the best combination for treatment. Secondly in the general AI information chapter we explained and presented Artificial Intelligence concepts, we talked about its 2 sub categories machine learning and deep learning and their characteristics and ended it talking about Computer Vision because it was the most used branch of AI in the discussed methods, we mentioned its implementation in both machine learning and deep learning and its famous applications.

Reaching the state of the art, we did a recap and summary of used methods explaining their line of reasoning, datasets used, issues they faced, different feature extraction techniques and their algorithms with their evaluation metrics. Since there is a lot of work in this field we firstly presented articles that implement the usual method (image -$>$ process -$>$ train -$>$ predict) then secondly concentrated on papers that presented new ideas such as using a different type of dataset (such as Raman spectroscopy or the combination of images and clinical information) and finally we presented a comparison table of said methods and added 2 more tables from previous researchers that did a comprehensive survey of literature of the last few years.

After the recap, we discussed the differences and presented and extracted useful information that could help future researchers and contributors to have an overview and a ``map" of this field before diving into it. 