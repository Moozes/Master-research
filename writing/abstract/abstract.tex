\section*{Abstract}
skin caner is one of the most common cancers in the world and it can be fatal if not treated early, that is why its early diagnosis is considered to be the best treatment for it. and under the light of reacent advancements in computational power and in the artificial intelligence field (especialy its 2 subdomains machine learning and deep learning) C.A.D (computer aided diagnosis) is considered to be one the best ways for early skin cancer diagnosis. that is why in this article we are going to do a comparative study of recent methods and algorithms applied in skin cancer analysis, detection and classification our comparaison is going to be based of different types of datasets used for trainning , different algorithms applied and famous performance metrics calculated by researchers such as accuracy, specificity, AUC (area under curve) ...etc in the hopes of better understanding the problem at hand and its applied solutions, and understanding some new explored ideas and challenges faced by researchers and contributors and finaly this article will help new researchers to understand what is ahead of them before engaging and contributing to this field.



% note idont need to do benchark testing because almost all these studies were applied on standard benchmark datasets such as ISIC 