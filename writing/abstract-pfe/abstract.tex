\section*{Abstract}
skin caner is one of the most common cancers in the world and it can be fatal if not treated early, that is why its early diagnosis is considered to be the best treatment for it. and under the light of reacent advancements in computational power and in the artificial intelligence field (especialy its 2 subdomains machine learning and deep learning) C.A.D (computer aided diagnosis) is considered to be one the best ways for early skin cancer diagnosis. that is why in this article we are going to do a comparative study of recent methods and algorithms applied in skin cancer analysis, detection and classification our comparaison is going to be based of different types of datasets used for trainning , different algorithms applied and famous performance metrics calculated by researchers such as accuracy, specificity, AUC (area under curve) ...etc in the hopes of better understanding the problem at hand and its applied solutions, and understanding some new explored ideas and challenges faced by researchers and contributors and finaly this article will help new researchers to understand what is ahead of them before engaging and contributing to this field.

\begin{center}
    \RL{الملخص}
\end{center}
\begin{RLtext}

    يعد سرطان الجلد أحد أكثر أنواع السرطان شيوعًا في العالم ويمكن أن يكون قاتلًا إذا لم يتم علاجه مبكرًا ، ولهذا السبب يعتبر التشخيص المبكر له هو أفضل علاج له. وتحت ضوء تطورات المفاعلات في القوة الحسابية وفي مجال الذكاء الاصطناعي (ولا سيما المجالين الفرعيين للتعلم الآلي والتعلم العميق) ، يُعد CAD (التشخيص بمساعدة الكمبيوتر) أحد أفضل الطرق للتشخيص المبكر لسرطان الجلد. هذا هو السبب في أننا سنقوم في هذه المقالة بدراسة مقارنة للأساليب والخوارزميات الحديثة المطبقة في تحليل سرطان الجلد واكتشافه وتصنيفه ، وستستند مقارنتنا إلى أنواع مختلفة من مجموعات البيانات المستخدمة للتدريب ، وخوارزميات مختلفة مطبقة وأداء مشهور المقاييس التي يحسبها الباحثون مثل الدقة والنوعية والجامعة الأمريكية بالقاهرة (المنطقة تحت المنحنى) ... إلخ على أمل فهم أفضل للمشكلة المطروحة وحلولها التطبيقية ، وفهم بعض الأفكار الجديدة المستكشفة والتحديات التي يواجهها الباحثون والمساهمون ونهاية ستساعد هذه المقالة الباحثين الجدد على فهم ما ينتظرهم قبل الانخراط في هذا المجال والمساهمة فيه.


\end{RLtext}



\section*{Resumé}
Le cancer de la peau est l'un des cancers les plus répandus dans le monde et il peut être mortel s'il n'est pas traité tôt, c'est pourquoi son diagnostic précoce est considéré comme le meilleur traitement. et à la lumière des avancées récentes en matière de puissance de calcul et dans le domaine de l'intelligence artificielle (en particulier ses 2 sous-domaines d'apprentissage automatique et d'apprentissage en profondeur), la C.A.D (diagnostic assisté par ordinateur) est considérée comme l'un des meilleurs moyens de diagnostic précoce du cancer de la peau. c'est pourquoi, dans cet article, nous allons faire une étude comparative des méthodes et algorithmes récents appliqués à l'analyse, à la détection et à la classification du cancer de la peau. Notre comparaison va être basée sur différents types d'ensembles de données utilisés pour l'entraînement, différents algorithmes appliqués et performances célèbres. métriques calculées par les chercheurs telles que la précision, la spécificité, l'AUC (aire sous la courbe) ... etc dans l'espoir de mieux comprendre le problème à résoudre et ses solutions appliquées, et de comprendre certaines nouvelles idées explorées et les défis auxquels sont confrontés les chercheurs et les contributeurs et enfin cet article aidera les nouveaux chercheurs à comprendre ce qui les attend avant de s'engager et de contribuer à ce domaine.
% note idont need to do benchark testing because almost all these studies were applied on standard benchmark datasets such as ISIC 

% previous esi memoires didnt use benchmarks, and they compared using accuracy! only
% and the other momoire they brought the comparaison of others and included it in there article!

% AUC the probability to chose right class