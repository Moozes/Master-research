\section{Skin Cancer}

        Skin cancer is the abnormal growth of cells found in the epidermis (the outer layer of the skin)~\cite{scf2022}, it is one of the most common cancers in the world ~\cite{nhs2020} and it falls under the category of a malignant tumor that is formed by fast multiplication of cells wich is caused by mutations/damage in the DNA  of those cells, the damage in there DNA is due to the exposure to ultra violet rays ~\cite{scf2022} which can come from various sources but the most common are sun light and tanning beds [figure tannig bed] ~\cite{mayo2020, scf2022, nhs2020}. The most common types of skin cancer are  basal cell carcinoma, squamous cell carcinoma , melanoma. The good news is that if it is descovered in an early stage or pre cancurous stage it can be treated easily without leaving a scar.

    %--------tannig bed and sunlamps---------


\subsection{Symptoms} 
        Skin cancer can appear in any place on the body that is exposed to sunlight like : face, scalp, chest ...etc, but there are some cases where the cancer appeared in areas not always exposed to sunlight such as  palm, soles, under the finger neils ~\cite{mayo2020}.
        Skin cancer can happen to people of any skin color but it is know that people with darcker skins are less likely to have it because of the protection against ultra violet rays provided by the melanin which present in darcker people in more quantities than pale people ~\cite{mayo2020}.

        \begin{enumerate}
        \item Basal cell carcinoma signs and symptoms Figure ~\ref{fig:basal}
            \begin{itemize}
            \item bump
            \item flat brown scar
            \item bleeding sore that heals and returns
            \end{itemize}
        \item Squamous cell carcinoma signs and symptoms Figure ~\ref{fig:squamous}
            \begin{itemize}
            \item red nodule
            \item flat lesion with crusted surface
            \end{itemize}
        \item Melanoma signs and symptoms Figure ~\ref{fig:melanoma}
            \begin{itemize}
            \item brownish spot
            \item painful lesion that itches and burns
            \item dark lesion
            \end{itemize}
        
        \end{enumerate}
        
        %------figure of the three---------
        \begin{figure}[h]
        \centering
            \begin{subfigure}[b]{0.3\textwidth}
                \centering
                \includegraphics[scale=.5]{./chapter-01-general-medical-information/Melanoma.jpg}
                \caption{Melanoma ~\cite{wiki2022}}
                \label{fig:melanoma}
            \end{subfigure}
            \begin{subfigure}[b]{0.3\textwidth}
                \centering
                \includegraphics[scale=.3]{./chapter-01-general-medical-information/Squamous.png}
                \caption{Squamous ~\cite{wd}}
                \label{fig:squamous}
            \end{subfigure}
            \begin{subfigure}[b]{0.3\textwidth}
                \centering
                \includegraphics[scale=.5]{./chapter-01-general-medical-information/basal.jpeg}
                \caption{Basal ~\cite{epr2021}}
                \label{fig:basal}
            \end{subfigure}
        \caption{3 Most Common Types of Skin Cancer}
        \label{fig:3types}
        \end{figure}


    \subsection{Types}
        The 3 most common types are the following ~\cite{scf2022}
        \begin{description}
        \item[Basal cell carcinoma] \hfill \\
            The most common type with about 3.6 million new cases each year in the united states , if not treated early it can cause local destruction it can spread and in rare cases it is fatal.
        \item[Squamous cell carcinoma] \hfill \\
            The second most common type with about 1.8 million new cases in the united states each year, if not treated early it will spread and it is in some cases fatal (15000 deaths/year in the united states).
        \item[Melanoma] \hfill \\
            one of the most common types, by 2022 it is estimated that 197700 will appear in the united states although it is treatable if detected early it is considered to be the most dangerous among common types because of its death rates (7650 deaths projected for the united states in 2022).
        \end{description}



    \subsection{Causes}
        The most common and main cause of skin cancer is the exposure to ultra violet ~\cite{mayo2020, scf2022, nhs2020} radiations that can primarely be found in sun light and tanning beds, but there are some cases where the cancer appeard in areas not exposed to the sun like palms, soles, and under finger neils which indicates that other factors may contribute to the formation of skin cancer such as toxic substances, weak immune system, other types of radiation ...etc ~\cite{mayo2020}.
        The cells that the skin cancer originates from are squamous cells, basal cells and melanocytes. Squamous cells is just below the outer surface, basal cells is beneath squamous cells and it produces new skin cells and melanocytes are the cells responsible of generating melanin which is the pigment resposible of the skin color. ~\cite{mayo2020}.


    \subsection{Risk Factors}
        Factors that may increase your chances of getting skin cancer are ~\cite{mayo2020}
        \begin{description}
        \item[Fair skin] \hfill \\
            If you have less melanin which means your skin color is less dark then you are much more likely to get skin cancer then a person with a darcker skin because the melanin pigment is responsible of protecting the skin from ultra violet effects.
        \item[History of sun burn] \hfill \\
            Having had sun burns before either in childhood or adulthood may increase your chances.
        \item[Exposure to the sun for long periods of time] \hfill \\
            Being exposed to the sun alot or using tanning beds alot is also one of the factors, a tan is your skin's injury response of ultra violet rays. 
        \item[High altitude climates] \hfill \\
            Living in higher places like mountains means that you are exposed to strong sunlight.
        \item[Moles] \hfill \\
            Some types of irregular moles -which are bigger in size than normal moles- can turn cancerous.
        \item[Precancerous skin lesions] \hfill \\
            There are some types of skin lesions -which are in them selfs not cancerous- that are likely to turn cancerous such as Bowen's disease and  Actinic keratoses.
        \item[Family/Personal history of skin cancer] \hfill \\
        \item[Weak immune system] \hfill \\
            Such as having HIV, AIDS or taking immunosuppressant drugs after an organ transplant...etc.
        \item[Exposure to radiation] \hfill \\
        \item[Exposure to certain substances] \hfill \\
            Some harmful/unharmful substances can increase your chances such as arsenic.
        \end{description}



    \subsection{Prevention}
        As it is mentioned in ~\cite{mayo2020, scfp2022} 
        \begin{itemize}
        \item avoid the sun at the middle of the day
        \item use sunscreen to protect against sunburn with an spf (Sun Protection Factor) over 30
        \item protective clothing especially when living in the desert
        \item avoid tanning beds
        \item always check your body for abnormalities and report them to your doctor
        \item see a dermatologist at lest once a year
        \end{itemize}


    \subsection{Treatement}
        Before treatement we need diagnosis first, there are two methods ~\cite{clinic2020} to  know that you might have skin cancer. The first method is by observing you skin frequently to see if there are some marks or abnormalities, after that you check in with a doctor who will preform further examinations which will bring us to the second method, skin biopsy -taking a part of the suspecious area of the skin and preforming some laboratory tests on it to have accurate results-.
        After confirming that you have a skin cancer further tests will determin what stage is it at which is often refered to with Roman numbers (I means small and limited to the area where it started - IV means advanced cancer that has spread to other parts of the body).
        Treatement methods may vary depending on the size, type and stage of the cancer ~\cite{nhs2020} but the main way to treate cancer is to remove it completely especialy if it is in early or pre-cacerous stages otherwise if additional treatement is needed, the options are as mentioned in ~\cite{clinic2020}:
            \begin{itemize}
                \item freezing with liquid nitrogen
                \item Mohs surgery which is for difficult cases where the surrounding healthy skin cant be removed with cancerous cells (such as the nose area)
                \item Curettage and electrodesiccation to illiminate remaining cancerous cells  
                \item Radiation therapy such as X-rays
                \item chemotherapy with substances that contain anti caner properties such as lotions if the cancer is on the surface
                \item Photodynamic therapy, a combination of laser and chemicals 
                \item Biological therapy using the body's own immune system
            \end{itemize}

