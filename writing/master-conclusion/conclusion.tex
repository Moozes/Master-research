    Skin cancer is one the most dangerous and wide spread cancers in the world, it can occur to anyone of any age and of any race, just by standing in the sun fo too long will increase your chances of getting it. that is why an early diagnosis will save your life. here where the CADs (computer aided diagnosis) systems would play an importatnt role by implimenting machine learning algorithms that could recognise, detect and classify various skin lesions .

    The research in this field has brought alot of advancements both in machine learning and deep learning and has produced alot of models that reach an accuracy above that of an expert dermatologist but we cant do without the expertise of doctors because of the interpretability problem since most models and research contributions dont take into consideration this aspect of algorithms.

    There are alot of datasets available online, most of them are Missing some aspect or another but that can be solved combining different datasets while taking into consideration the imbalances and heteroginity of said datasets.

    In this article we did a recap of most famous methods, algorithms and datasets used and we did a little comparaison between them based on performance metrics, dataset caracteristics and applicability in real world scenarios, reading this article will help futur researchers who are planning on contributing in this field.
    
    We conclude that there are alot of things to take into considerations when building a mahine learning model, there is no right or wrong way but in the hopes of drawing a guidance map for futur contributers, we made a line of reasoning that could guide them in the right direction. 




