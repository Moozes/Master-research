Skin cancer is one of the most dangerous and widespread cancers in the world, it can occur to anyone of any age and of any race, just by standing in the sun for too long will increase your chances of getting it. That is why an early diagnosis will save your life. Here where the CAD's (computer aided diagnosis) systems would play an important role by implementing machine learning algorithms that could recognize, detect and classify various skin lesions.

The research in this field has brought a lot of advancements both in machine learning and deep learning and has produced a lot of models that reach an accuracy above that of an expert dermatologist, but we can't do without the expertise of doctors because of the interpretability problem since most models and research contributions don't take into consideration this aspect of algorithms.

There are a lot of datasets available online, most of them are missing some aspect or another, but that can be solved combining different datasets while taking into consideration the imbalances and heterogeneity of said datasets.

In this article we did a recap of most famous methods, algorithms and datasets used, and we did a little comparison between them based on performance metrics, dataset characteristics and applicability in real world scenarios, reading this article will help future researchers who are planning on contributing in this field.

We conclude that there are a lot of things to take into considerations when building a machine learning model, there is no right or wrong way but in the hopes of drawing a guidance map for future contributors, we made a line of reasoning that could guide them in the right direction. 




